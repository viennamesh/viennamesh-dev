\section{Usage}
Based on the provided input mesh in the folder 

\begin{exaipd}
\begin{Verbatim}
examples/
\end{Verbatim}
\end{exaipd}

the complete toolchain is introduced.

\subsection{Conversion}
First, the input file has to be converted to a file format 
which is used throughout the toolchain.
We consider that all executables are being present in the 
\begin{exaipd}
\begin{Verbatim}
bin/
\end{Verbatim}
\end{exaipd}
folder, and the working directory is the root folder of the package.

To convert the file, the following command has to be executed
\begin{exaipd}
\begin{Verbatim}
bin/hull_converter examples/device.hin device.gau32
\end{Verbatim}
\end{exaipd}

\subsection{Hull Orienter}
It may happen, that the hull mesh files are not oriented. This most 
probably results in a failing hull adaption processing step. 
This is exactly the case with the present input mesh. 
Therefore, prior to any further adaption or volume meshing steps, the mesh 
has to be oriented. 

\NOTE{The hull orientation utility is a prototype. Therefore it may introduce
problems on correct input meshes. Only use it, when the hull adaption 
process fails for a particular input mesh.}

This can be done by the following command.

\begin{exaipd}
\begin{Verbatim}
bin/hull_orienter device.gau32 device_oriented.gau32
\end{Verbatim}
\end{exaipd}

For further meshing steps, the oriented mesh shall be used.

\newpage

\subsection{Hull Adaption}
The converted, oriented hull mesh can now be processed by the hull mesh adaption 
tool. This utility increases the quality of the hull mesh, and by doing so, 
it significantly improves the quality of the generated volume mesh, which will 
be generated based on the adapted hull mesh.

\begin{exaipd}
\begin{Verbatim}
bin/hull_adaptor device_oriented.gau32
\end{Verbatim}
\end{exaipd}

Note, that this utility has no option for a specific output mesh name, it 
always produces output meshes with the filename

\begin{exaipd}
\begin{Verbatim}
surface_mesh.gau32
\end{Verbatim}
\end{exaipd}

Therefore, we rename the adapted output mesh to something like this

\begin{exaipd}
\begin{Verbatim}
mv surface_mesh.gau32 device_adapted.gau32
\end{Verbatim}
\end{exaipd}

\subsection{Volume Mesher}
The adapted hull mesh can now be volume meshed by the following command:

\begin{exaipd}
\begin{Verbatim}
bin/volume_mesher device_adapted.gau32 device_adapted.gau3
\end{Verbatim}
\end{exaipd}

\subsection{Mesh Classifier}
The resulting volume mesh can now be investigated regarding the quality 
of the mesh. Therefore a utility has been developed which evaluates 
the quality of a mesh~\cite{heinzlgen}~\cite{stimpflmulti}. 
This can be done by the following command:

\begin{exaipd}
\begin{Verbatim}
bin/mesh_classifier device_adapted.gau3
\end{Verbatim}
\end{exaipd}

The tool outputs statistics on the different mesh element types. 
In the following, the mesh element type statistics is presented of 
the volume mesh under consideration.

\begin{exaipd}
\begin{Verbatim}
---------------------------------------------
  cap       0.604686 %     <-- a small number is good
---------------------------------------------
  needle    0.748986 %     <-- a small number is good
---------------------------------------------
  round     80.0866 %      <-- a large number is good
---------------------------------------------
  slat      2.7898 %       <-- a small number is good
---------------------------------------------
  sliver    6.98138 %      <-- a small number is good
---------------------------------------------
  spade     1.30557 %      <-- a small number is good
---------------------------------------------
  spindle   0.542843 %     <-- a small number is good
---------------------------------------------
  wedge     6.94015 %      <-- a small number is good
---------------------------------------------
\end{Verbatim}
\end{exaipd}

Of those presented types, the \texttt{round} and the \texttt{sliver} 
are of particular importance. The \texttt{round} type is considered good, 
because the dihedral angles of the tetrahedron are in a good mid-range. 
Those types of elements are usually good for discretization schemes, like 
Finite Volume Methods or Finite Element Methods. 

On the contrary, a \texttt{sliver} tetrahedron is a highly degenerated one. 
Consequently, a small number is desirable. Aside from the \texttt{sliver}
element type, there are other degenerated elements. Obviously, the number of 
those should be as small as possible. 

Note, that the mesh classifier tool also produces two files, one containing 
the present dihedral angles, and the other contains Latex code. 
The file containing angles can be used, for example, to create histograms.
Therefore, the angle distribution throughout the whole mesh can be investigated.
The file containing Latex code, contains a table with the overview of different 
mesh element types and a histogram like figure representing the different mesh 
element types. 
Those outputs significantly support investigations of the mesh quality.

